\documentclass{article}
\usepackage[utf8]{inputenc}
\usepackage{listings}
\usepackage{xcolor}


\definecolor{codegreen}{rgb}{0,0.6,0}
\definecolor{codegray}{rgb}{0.5,0.5,0.5}
\definecolor{codepurple}{rgb}{0.58,0,0.82}
\definecolor{backcolour}{rgb}{0.95,0.95,0.92}

\lstdefinestyle{mystyle}{
    backgroundcolor=\color{backcolour},   
    commentstyle=\color{codegreen},
    keywordstyle=\color{magenta},
    numberstyle=\tiny\color{codegray},
    stringstyle=\color{codepurple},
    basicstyle=\ttfamily\footnotesize,
    breakatwhitespace=false,         
    breaklines=true,                 
    captionpos=b,                    
    keepspaces=true,                 
    numbers=left,                    
    numbersep=5pt,                  
    showspaces=false,                
    showstringspaces=false,
    showtabs=false,                  
    tabsize=2
}

\lstset{style=mystyle}

\begin{document}

\title{The Longest Path}
\author{Vu Minh Hien Bi12-154}
\date{\today}

\maketitle

\section{Explanation of Mapper and Reducer}

\subsection{Mapper (find\_files function)}
The \texttt{find\_files} function takes two parameters: \texttt{file\_name} (the name of the file to search for) and \texttt{search\_path} (the directory path to start the search from). It works as follows:
\begin{itemize}
    \item It initializes an empty list \texttt{found\_file\_paths} to store the paths of the found files.
    \item It then iterates through each directory and subdirectory starting from \texttt{search\_path} using \texttt{os.walk()}.
    \item For each directory, it checks all the files within that directory.
    \item If a file with the name \texttt{file\_name} is found, its full path is appended to the \texttt{found\_file\_paths} list.
    \item Finally, it returns the list of found file paths.
\end{itemize}

\subsection{Reducer}
In the context of your code, the reducer can be considered as the part of the code that follows the mapper function. After calling \texttt{find\_files}, the code checks if any files were found.

\section{Results}

Here are the results:





\section{The Longest Path}
\begin{lstlisting}[language=C++, caption=The Longest Path C++ Code, label=lst:code]
import os

def find_files(file_name, search_path="C:/"):
    found_file_paths = []
    for root, dirs, files in os.walk(search_path):
        for file in files:
            if file == file_name:
                found_file_paths.append(os.path.join(root, file))
    return found_file_paths


file_name_to_find = input("Enter the file name to find: ")
found_file_paths = find_files(file_name_to_find)

if found_file_paths:
    print("File(s) found:")
    for file_path in found_file_paths:
        file_path = file_path.replace("\\", "/")
        print(file_path)
else:
    print("No file(s) found.")

mapped_data = [
    (path.split(os.sep), len(path.split(os.sep))) for path in found_file_paths
]

longest_path = max(mapped_data, key=lambda x: x[1])

print("Longest path(s):", "/".join(longest_path[0]))
\end{lstlisting}


\end{document}