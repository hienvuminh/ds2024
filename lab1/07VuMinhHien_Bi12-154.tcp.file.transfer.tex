\documentclass{article}
\usepackage[utf8]{inputenc}
\usepackage{listings}
\usepackage{xcolor}


\definecolor{codegreen}{rgb}{0,0.6,0}
\definecolor{codegray}{rgb}{0.5,0.5,0.5}
\definecolor{codepurple}{rgb}{0.58,0,0.82}
\definecolor{backcolour}{rgb}{0.95,0.95,0.92}

\lstdefinestyle{mystyle}{
    backgroundcolor=\color{backcolour},   
    commentstyle=\color{codegreen},
    keywordstyle=\color{magenta},
    numberstyle=\tiny\color{codegray},
    stringstyle=\color{codepurple},
    basicstyle=\ttfamily\footnotesize,
    breakatwhitespace=false,         
    breaklines=true,                 
    captionpos=b,                    
    keepspaces=true,                 
    numbers=left,                    
    numbersep=5pt,                  
    showspaces=false,                
    showstringspaces=false,
    showtabs=false,                  
    tabsize=2
}

\lstset{style=mystyle}

\begin{document}

\title{GAE}
\author{Vu Minh Hien Bi12-154}
\date{\today}

\maketitle


\maketitle



\section{Important snippets}
\begin{enumerate}
    \item \textbf{Initialization of the Sudoku board:} The \texttt{\_\_init\_\_} method initializes the Sudoku board with a $9 \times 9$ grid filled with zeros.

    \item \textbf{Printing the board:} The \texttt{print\_board} method prints the current state of the Sudoku board in a visually appealing format with appropriate row and column separators.

    \item \textbf{Validity checks for moves:} Methods like \texttt{is\_valid\_row}, \texttt{is\_valid\_col}, and \texttt{is\_valid\_subgrid} check whether a move is valid based on Sudoku rules.

    \item \textbf{Main game loop:} The \texttt{main} function sets up the game, generates a full Sudoku board, removes a certain number of cells to create the puzzle, and then enters a loop where the player can input their moves until the puzzle is solved.
\end{enumerate}

\section{How structure code}
\begin{enumerate}
    \item \textbf{Class Definition:} The code defines a \texttt{Sudoku} class, encapsulating all functionalities related to the Sudoku game.

    \item \textbf{Methods for Generating and Solving Sudoku:} The \texttt{generate}, \texttt{solve}, and \texttt{generate\_full\_board} methods are placeholders for functionalities related to generating and solving Sudoku puzzles. Currently, they are empty and need to be implemented.

    \item \textbf{Input and Output Handling:} The \texttt{print\_board} method handles the visual representation of the Sudoku board, while the \texttt{main} function takes user input for row, column, and number, checks the validity of the move, updates the board accordingly, and prints the updated board.

    \item \textbf{Game Loop:} The main game loop in the \texttt{main} function keeps the game running until the puzzle is solved. Within this loop, the user is prompted to input their moves, and the board is updated accordingly.

    \item \textbf{Conditional Check for Puzzle Completion:} The code checks if there are no zeros left on the board to determine if the puzzle has been solved, indicating that all cells have been filled correctly.
\end{enumerate}

\section{Code}
\begin{lstlisting}[language=C++, caption=GAE C++ Code, label=lst:code]


import random
import numpy as np
from copy import deepcopy

class Sudoku:
    def __init__(self):
        self.board = [[0]*9 for _ in range(9)]

    def generate(self):
        pass 

    def print_board(self):
        for i in range(9):
            if i % 3 == 0 and i != 0:
                print("- " * 11)
            for j in range(9):
                if j % 3 == 0 and j != 0:
                    print("|", end=" ")
                print(self.board[i][j], end=" ")
            print()

    def is_valid_move(self, row, col, num):
        return self.is_valid_row(row, num) and self.is_valid_col(col, num) and self.is_valid_subgrid(row - row % 3, col - col % 3, num)

    def is_valid_row(self, row, num):
        return num not in self.board[row]

    def is_valid_col(self, col, num):
        return all(row[col] != num for row in self.board)

    def is_valid_subgrid(self, row, col, num):
        for i in range(3):
            for j in range(3):
                if self.board[i + row][j + col] == num:
                    return False
        return True

    def solve(self):
        pass  

    def generate_full_board(self):
        pass  

    def remove_cells(self, num_cells_to_remove):
        pass  

def main():
    game = Sudoku()
    game.generate_full_board()
    game.remove_cells(50) 
    game.print_board()

    while True:
        row = int(input("Enter row (1-9): ")) - 1
        col = int(input("Enter column (1-9): ")) - 1
        num = int(input("Enter number (1-9): "))

        if game.is_valid_move(row, col, num):
            game.board[row][col] = num
            game.print_board()
        else:
            print("Invalid move. Try again.")

        if all(0 not in row for row in game.board):
            print("Congratulations! You solved the puzzle   .")
            break

if __name__ == "__main__":
    main()

\end{lstlisting}

\end{document}